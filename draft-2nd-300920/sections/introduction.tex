\chapter{Introduction}

 The research of this dissertation is conducted at the Large Hadron Collider (LHC) [ref] with the ATLAS experiment [ref] using full data-set collected 
 during 2015-2018 in proton-proton collisions at a center-of-mass energy of 13 TeV.  The total integrated luminosity of the data-set is 139 fb$^{-1}$. 
 Two physics analysis topics are presented in this thesis. The author of this thesis has played leading roles and made significant contributions in these analyses.
 
 The first topic is a search for the Standard Model (SM) Higgs boson decay to di-muon final state to measure the Higgs Yukawa coupling to the second generation of fermions. 
 Unlike the universal gauge couplings introduced by the gauge invariant principle in construction of the theory frame, the Yukawa couplings are not universal, depending on the masses of fermions, which is a new type of coupling representing a new type of force related to the Higgs boson which generates masses for all the elementary particles. Compared to the third generation fermions, tau lepton, top and bottom quarks, the masses of the second generation fermions are much light. For example, the mass of the tau lepton (in the third generation) is 1.78 GeV, while the mass of the muon (in the second generation) is only 106 MeV, as the consequence the coupling of the Higgs boson to the di-muon is about 17 times smaller and the event rate is about 280 smaller compared to the Higgs boson decay to di-tau final state. The Higgs boson decay to the third generation of fermions have been observed at the LHC [ref]. In recent years the focus of the experiments, ATLAS and CMS [ref] at the LHC, is to observe the Higgs boson decay to muon or charm-quark pairs. Due to experimental difficulties in identifying the charm-quarks from huge background the di-muon channel is perhaps the only feasible channel for the second generation coupling measurement at the LHC. 
 Recently both ATLAS and CSM experiments announced that the Higgs boson decay to di-muon signals have been detected with statistical significance of 2 - 3$\sigma$ [ref]. This dissertation will describe the physics analysis in ATLAS in searching for the Standard Model (SM) Higgs boson decays to di-muon, a very rare Higgs decay process.
 
 The other topic of this thesis is searching for new physics using the dilepton (dimuon and dielectron) invariant mass spectra 
 to probe quark and lepton internal structures to address the most fundamental question in particle physics research. 
 The physics benchmark in the search is the contact-interaction between quarks and leptons through the processes of di-lepton pair production 
 from quark-antiquark annihilation. The experimental signature would be non-resonant event enhancement in multi-TeV di-lepton mass scale. 
 Compared to resonant search, which would have a well-defined resonant peak in the mass spectra, the non-resonant signature in the high mass spectra tails 
 would be much hard to detect due to large statistical uncertainty as well as large theoretical and experimental uncertainties. 
 In previous ATLAS analysis, background estimation was made by using the Monte Carlo simulated events which introduced large uncertainties 
 in the theory modeling, such as the uncertainty in high mass range due to the parton-density function of proton used in theoretical calculations and simulations.  
 A novel approach is developed in this thesis work
 by using data in the background control regions (background enriched phase space) to estimate the background in the signal regions, 
 which greatly reduced background modeling uncertainties. Many computer simulated “experiments” were performed to optimize the boundaries 
 between the background control regions and the signal regions with different final states and different theoretical models.
 No significant data excess over the background expectation, therefore the lower limits on the energy scale, 38.5 TeV, of the contact-interaction is set, 
 indicating that the quarks and leptons are still point-like particles at 10$^{-20}$ m, 
 this is the strongest limit on $q\bar q \ell^+\ell^-$ contact-interactions to date.
 The results of this search has been recently published [ref] by the ATLAS Collaboration.
 
This thesis is written in nine chapters, in addition to the brief introduction chapter,
Chapter 2 introduces the SM theory of particle physics;
Chapter 3 describes the LHC and the ATLAS experiment;
Chapter 4 describes the phenomenology related to the proton-proton collisions that related to this thesis physics analysis;
Chapter 5 reports the data-sets, including data from the LHC and Monte Carlo simulated events, used in the analyses;
Chapter 6 presents the analysis and results of searching for the SM Higgs decay to di-muons;
Chapter 7 presents the analysis and results of searching for non-resonance signature of the contact-interactions;
Chapter 8 summarize the research results of this thesis;
Chapter 9 presents the future prospects at the LHC in this thesis related physics topics.
 
%\subsection{Acknowledgements}
%\begin{itemize}
%    \item Bing
%    \item Matt
%    \item Noam
%    \item Deshan
%    \item Dilepton team
%   \item EB
%    \item Bing Li and Yanlin
%    \item QFT and Particle teachers
%    \item Root Forum posters
%    \item Officemates
%    \item Secretaries
%    \item Michigan professors
%    \item Engineers
%    \item CERN/LHC/ATLAS
%\end{itemize}

% History
