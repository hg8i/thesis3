% ##############################################################
\section{The Universe and the Atom}
% ##############################################################

The Universe consists of particles, the space they inhabit, and the interactions within and between these.
The extent to which these components are described by mathematics is the profound, and historically unintuitive, foundation on which the field of physics is based.
This chapter provides the mathematical context for this description.

To make this chapter self contained, this section will first introduce a variety of topics related to particle physics.
These topics are introduced in a general way with names that are intuitively understood.
They will later be revisited to provide mathematically precise definitions in the following section.
After this, the Standard Model of particle physics (SM) will be described.
Finally, two sections will present interpretation of the SM in the context of the work of this thesis. First for Higgs coupling to muons, and next for contact interactions.

\subsection{Particles}
Fundamental particles are the most basic building block of matter.
They are defined by several properties including their mass, their charge, and their spin.
The (SM) identifies a number of fundamental particles that appear as mathematical objects within the theory.
Particles with integer spins (0, 1, 2) are called Bosons, while particles with half-integer spins (\half, -\half) are called Fermions.

% Fermions
The Fermions are further divided into quarks with fractional electric charge, and leptons with integer electric charge.
Quarks commonly exist in bound states consisting of two (meson) and three (baryon) quarks.
The charged leptons, in order of ascending mass, are the electron, the muon, and the tauon.
The neutral leptons are neutrinos.
% Bosons
The Bosons mediate forces between the other particles.
The most familiar boson, the photon, mediates the electromagnetic force.
The gluon carries the strong nuclear force.
The weak nuclear force is carried by the \W and \Z bosons.
Finally the recently discovered Higgs boson mediates a momentum exchange between particles.

% Summary
These particles are summarized in Table \ref{tab:particles}.

\begin{table}[htp]
\begin{center}
\caption{Particle of the Standard Model listed along with their symbol and several properties.}
{\footnotesize
\begin{tabular}{c c l c c c c c c c}
\toprule
& & Name & Symbol & Charge & Spin & Mass [MeV/c$^2$] \\
\midrule
\multirow{12}{*}[0em]{\begin{sideways}Fermions\end{sideways}} & \multirow{6}{*}[0em]{\begin{sideways}Leptons\end{sideways}} & Electron & \e & -1 & 1/2 & 0.511 \\
& & Muon   			  & \m         & -1 & 1/2 & 105.7 \\
& & Tau    			  & $\tau$     & -1 & 1/2 & 1776.8 \\
& & Electron Neutrino & $\nu_e$    & 0  & 1/2 &  $<2\times10^{-6}$\\
& & Muon Neutrino     & $\nu_\mu$  & 0  & 1/2 &  $<2\times10^{-6}$\\
& & Tau Neutrino      & $\nu_\tau$ & 0  & 1/2 &  $<2\times10^{-6}$\\
\cline{2-7} 
& \multirow{6}{*}[0em]{\begin{sideways}Quarks\end{sideways}} & Up & $u$ & 2/3 & 1/2 &  $2.2\pm0.5$ \\
& & Charm             & $c$ &  2/3 & 1/2 &  $1.275\pm0.035\times10^{3}$ \\
& & Top               & $t$ &  2/3 & 1/2 &  $173.0\pm0.4\times10^{3}$ \\
& & Down              & $d$ & -1/3 & 1/2 &  $4.7\pm0.5$ \\
& & Strange           & $s$ & -1/3 & 1/2 &  $95\pm9$ \\
& & Bottom            & $b$ & -1/3 & 1/2 &  $4.18\pm0.04\times10^{3}$ \\
\midrule
\multicolumn{2}{c}{\multirow{6}{*}[0em]{\begin{sideways}Bosons\end{sideways}}} & Photon & $\gamma$ & 0 & 1 & $<1\times10^{-24}$ \\
& & Gluon   & $g$ & 0 & 1 & 0 \\
& & Z       & \Z  & 0 & 1 & $91.1876\times 10^3$ \\
& & W       & \Wp & 1 & 1 & $80.39\times 10^3$ \\
& & W       & \Wm &-1 & 1 & $80.39\times 10^3$ \\
& & Higgs   & \h  & 0 & 0 & $125.18\times 10^3$ \\
& & Graviton& $g$ & 0 & 2 & $<1\times 10^{-38}$ \\
\bottomrule
\end{tabular}
}
\label{tab:particles}
\end{center}
\end{table}

\subsection{Space}

\subsection{Interactions}
Until the discovery of the Higgs boson, there were four known forces through which particles might interact: gravity, the electromagnetic force, the weak nuclear force, and the strong nuclear force.
An interaction between two or more particles entails the exchange of momentum between the participants.
For each force, the momentum exchange is mediated by a boson.
The strong nuclear force is mediated by gluons.
The weak nuclear force is mediated by the \Wp, \Wm, and \Z bosons.
The electromagnetic force is mediated by photons.
It is expected that gravity is mediated by a hypothetical particle called a graviton, however this has not been observed.

In 2012, the ATLAS and CMS experiments at the Large Hadron Collider at CERN discovered the Higgs boson.
The Higgs boson, like the bosons associated with the four canonical forces, mediates a momentum exchange between particles.

\begin{table}[htp]
\begin{center}
\caption{Interactions in ($^*$and not in) the Standard Model. \cite{robinson}}
{\footnotesize
\begin{tabular}{c c l c c c c c c c}
\toprule
Interaction      & Carried By &  Strength   & Range [m]  \\
\midrule
Strong           & Gluon      &  1          & $10^{-15}$ \\
Electromagnetic  & Photon     &  $10^{-2}$  & $\infty$   \\
Weak             & \W, \Z     &  $10^{-5}$  & $10^{-18}$ \\
Gravity$^*$      & Graviton   &  $10^{-39}$ & $\infty$   \\
Higgs            & Higgs      &  $10^{-xx}$ &    \\
\bottomrule      
\end{tabular}
}
\label{tab:forces}
\end{center}
\end{table}
