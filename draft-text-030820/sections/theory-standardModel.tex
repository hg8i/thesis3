\subsection{The Standard Model}

The Standard Model of Particle Physics is the most descriptive and predictive theory of fundamental particles.
The theory is formulated in the terms described in Section \ref{sec:smMath}.
It is a theory whose components are operator valued fields.
These fields act as representations of the group $\text(SU)(3)\otimes\text{SU}(2)\otimes\text{U}(1)$.
This section will define the Standard Model in pieces, which will then be combined to yield the full theory.
\begin{itemize}
    \item First, Quantum Electrodynamics, the theory of electromagnetic interactions between the photon and electrically charged fermions.
    \item Next, Quantum Chromodynamics, the theory of the strong interaction mediated by gluons and exchanging momentum between color charged particles.
    \item Finally, the Electroweak Interaction, the combination of weak interactions mediated by the \W and \Z bosons with the electromagnetic interaction.
\end{itemize}

\subsubsection{Quantum Electrodynamics}

Quantum Electrodynamics, or QED, is a field theory that describes the interactions between photons and electrically charged fermions.
It is the simplest of the field theories which build up the standard model.
The theory predictions of the theory are invariant under the \poincare group transformations.
In addition, the predictions are also invariant under U(1) transformations, parameterized by a continuous angle $\theta$.

There are two components of the theory: photons and fermion fields.
Photons are identified as elements of the U(1) algebra, $A_\mu$.
Meanwhile fermion fields $\Psi$ and $\overline{\Psi}$, with electric charge $Q$ are identified as elements of the module on which those generators may act.
Both of these objects transform under elements of the U(1) representation:
\begin{equation}\begin{split}
    \Psi\to& e^{iQ\theta}\Psi \\
    A_\mu\to& A_\mu-\frac{1}{e}\partial_\mu\theta \\
\end{split}\end{equation}
In the first line, a unitary element of U(1) is represented by a complex scalar $e^{iQ\theta}$.
This is the fundamental representation for U(1). \check
This element can act on the 4-vector $\Psi$, which is an element of the representation's module.
% The fermion field $\Psi$ has four components, but it transforms under U(1) 
In the second line, the gauge field $A_\mu$, represented by $4\times4$ matrices, transforms under the adjoint representation of U(1) as a connection.
The reason for this transformation are provided in Appendix \ref{sec:appendixVectorField}.

These fields can be combined to produce a Lagrangian consisting of a free component, and an interaction term.
An example is shown in Equation \ref{eqn:qedLagrangian} with one photon field, and one fermion field.
This is readily generalizable to models with more charged fermion fields, or indeed more photon fields. 
\begin{flalign}\label{eqn:qedLagrangian}
\mathcal{L}_\text{0}=&-\frac{1}{4}F^{\mu\nu}F_{\mu\nu}+\overline{\Psi}(i\gamma^\mu\partial_\mu-m)\Psi; & \text{free Lagrangian} \notag\\
\mathcal{L}_\text{int}=&-eQ\overline{\Psi}\gamma^\mu\Psi A_\mu; & \text{interaction Lagrangian} \notag\\
\mathcal{L}=&\mathcal{L}_\text{0}+\mathcal{L}_\text{int}; & \text{full Lagrangian} 
\end{flalign}
Here $J^\mu=Q\overline{\Psi}\gamma^\mu\Psi$ is the electric current density $J^\mu=(\rho,\vec{J})$.
The object $F_{\mu\nu}=\partial_\mu A_\nu-\partial_\nu A_\mu$ is the \emph{electromagnetic field tensor}.
The charge of the electron, $e$, is the (energy dependant running) coupling.
It is related to the fine structure constant $\alpha\equiv\frac{e^2}{4\pi}=1/137...$.

Although the fields transform under elements of the group U(1), physical predictions of the theory (the Lagrangian) are invariant under these transformations.
To use the Language of group theory, the Lagrangian is a singlet, described by a trivial representation of the group. 
This means the parameter $\theta$ is free to be selected for the convenience of calculation. 
A common gauge choice selects $\theta(x^\mu)$ to be a function of spacetime such that $\partial_\mu A^\mu=0$.\footnote{Although the contracted indices make this gauge choice Lorentz invariant, it is called the Lorenz gauge after Ludvig Lorenz. The two physicists independently developed the Lorentz-Lorenz equation.}
\check

The fields $A_\mu$, $\Psi$, and $\overline{\Psi}$ can be decomposed in terms of creation/annihilated operators as shown in Equation \ref{eqn:qedFields}.
\begin{equation}\begin{split}\label{eqn:qedFields}
A_\mu=&\sum_{\lambda=0}^3\int d\tilde{p}[\epsilon_\mu(p,\lambda)e^{i\vec{p}\cdot\vec{x}}a_{\vec{p},\lambda}+\epsilon^*_\mu(p,\lambda)e^{-i\vec{p}\cdot\vec{x}}a^\dagger_{\vec{p},\lambda}] \\
\Psi(\vec{x})=&\sum_{s=1}^2\int d\tilde{p}[u(p,s)e^{i\vec{p}\cdot\vec{x}}b_{\vec{p},s}+v(p,s)e^{-i\vec{p}\cdot\vec{x}}d^\dagger_{\vec{p},s}] \\
\overline{\Psi}(\vec{x})=&\sum_{s=1}^2\int d\tilde{p}[\overline{u}(p,s)e^{-i\vec{p}\cdot\vec{x}}b^\dagger_{\vec{p},s}+\overline{v}(p,s)e^{i\vec{p}\cdot\vec{x}}d_{\vec{p},s}] \\
\end{split}\end{equation}
Where $u(p,s)$, $v(p,s)$, $\overline{u}(p,s)$, $\overline{v}(p,s)$ are a basis for solutions to the Dirac equation in Table \ref{tab:fields}.
The creation and annihilation operators are defined:  \\
    \begin{center}
        \begin{tabular}{l l}\toprule
        $d_{\vec{p},s}^\dagger$ & Anti-fermion creator operator \\
        $d_{\vec{p},s}$ & Anti-fermion annihilation operator    \\
        $b_{\vec{p},s}^\dagger$ & Fermion creator operator \\
        $b_{\vec{p},s}$ & Fermion annihilation operator    \\
        \bottomrule\end{tabular} \\
    \end{center} 
The creation operators add the appropriate particle to a state they act on, while the annihilation operators remove a particle.
For both fermions and anti-fermions, the annihilation operator acting on the vacuum state is equal to zero: $d_{\vec{p},s}\ket{0}=0$ and $b_{\vec{p},s}\ket{0}=0$.
These have anti-commutation relationships \mbox{$\{b_{\vec{p},s},b_{\vec{k},r}^\dagger\}=\{d_{\vec{p},s},d_{\vec{k},r}^\dagger\}=(2\pi)^32E_{\vec{p}}\delta^{(3)}(\vec{p}-\vec{k})\delta_{sr}$}.

The polarization of the photon field is described by orthonormal 4-vectors:
\begin{equation}\begin{split}
\epsilon^\mu(p,\lambda)\epsilon^*_\mu(p,\lambda') = \begin{cases}
    +1 &\lambda=\lambda'=0\\
    -1 &\lambda=\lambda'\in[1,2,3]\\
     0 &\lambda\ne\lambda'\\
\end{cases}
\end{split}\end{equation}
A gauge choice can be made such that $\epsilon^\mu=(0,\vec{\epsilon})$. 
Since the polarization is orthogonal to momentum 3-vector, $p_\mu\epsilon^\mu=0$.

Finally, the Lagrangian in Equation \ref{eqn:qedLagrangian} is typically simplified by introducing the covariant derivative $D_\mu\equiv\partial_\mu+iQeA_\mu$.
\begin{equation}\begin{split}\label{eqn:qedLagrangian2}
\mathcal{L}=&-\frac{1}{4}F^{\mu\nu}F_{\mu\nu}+i\overline{\Psi}\slashed{D}\Psi-m\overline{\Psi}\Psi; \quad\text{Lagrangian}
\end{split}\end{equation}
The material in Equation \ref{eqn:qedLagrangian2}, and the expansions of the fields in Equations \ref{eqn:qedFields}, along with the creation/annihilation operator anti-commutators, is sufficient to calculate predictions of the model.
However, it is very helpful to introduce Feynman rules to represent this.
This follows the prescription from Section \ref{sec:feynmanRules} and is summarized in Table \ref{tab:qedRules}.

% First, consider the interaction term $-eQ\overline{\Psi}\gamma^\mu\Psi A_\mu$ in Equation \ref{eqn:qedLagrangian}.



\begin{table}[htp]
\begin{center}
{\footnotesize
\begin{tabular}{l | l | l l l}
\toprule
Name & Lagrangian Term & Drawing & Rule \\
\midrule
Photon-fermion interaction & $-eQ\overline{\Psi}\gamma^\mu\Psi A_\mu$ &  $\feynmandiagram [small,baseline=(d.base), horizontal=d to b] { a -- [fermion] b -- [fermion] c, b -- [boson] d [particle=\(\gamma\)], };$ & $-ieQ\gamma^\mu$  \\[1.0em]
Photon propagator & $-\frac{1}{4}F^{\mu\nu}F_{\mu\nu}$ &  $\feynmandiagram [small,baseline=(d.base), horizontal=d to b] {b -- [boson] d , };$ & $\displaystyle\frac{i}{p^2+i\epsilon}\left[-g_{\mu\nu}+(1-\xi)\frac{p_\mu p_\nu}{p^2}\right]$  \\[1.0em]
Fermion propagator & $\overline{\Psi}(i\gamma^\mu\partial_\mu-m)\Psi$ &  $\feynmandiagram [small,baseline=(d.base), horizontal=b to d] {b -- [fermion] d , };$ & $\displaystyle\frac{i(\slashed{p}+m)}{p^2-m^2+i\epsilon}$  \\[1.0em]
Initial state fermion      & & $\feynmandiagram [small,baseline=(d.base), horizontal=b to d] {b -- [fermion] d [blob]};$ & $u(p,s)$ \\[1.0em]
Initial state anti-fermion & & $\feynmandiagram [small,baseline=(d.base), horizontal=d to b] {b[blob] -- [fermion] d };$ & $\overline{v}(p,s)$ \\[1.0em]
Final state anti-fermion   & & $\feynmandiagram [small,baseline=(d.base), horizontal=d to b] {b -- [fermion] d [blob]};$ & $\overline{u}(p,s)$ \\[1.0em]
Final state fermion        & & $\feynmandiagram [small,baseline=(d.base), horizontal=b to d] {b[blob] -- [fermion] d };$ & $v(p,s)$ \\[1.0em]
Initial state photon       & & $\feynmandiagram [small,baseline=(d.base), horizontal=b to d] {b -- [boson] d [blob]};$ & $\epsilon_\mu(p,\lambda)$ \\[1.0em]
Final state photon         & & $\feynmandiagram [small,baseline=(d.base), horizontal=b to d] {b[blob] -- [boson] d };$ & $\epsilon^*_\mu(p,\lambda)$ \\[1.0em]
\bottomrule
\end{tabular}
}
\caption{Summary of the Feynman rules for QED. For each loop in the diagram with undefined momentum, an integration over the momenta contributes to the matrix element: $\int \frac{d^4k}{(4\pi)^4}$. Each closed fermion loop contributes a $-1$ factor, as does each odd permutation of the spinors $u$, $v$, $\overline{u}$, and $\overline{v}$. If there is ambiguity such that placement of the internal lines can have $N$ permutations, a factor or $1/N$ is added to remove double counting. The number $\xi$ is free to choose as a guage choice, and $\epsilon$ in the propagators is infinitesimal.}
\label{tab:qedRules}
\end{center}
\end{table}

