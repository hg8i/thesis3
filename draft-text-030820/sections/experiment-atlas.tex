\section{ATLAS Detector}
%ATLAS detector
There are seven experiments around the circumference of the LHC located at the interaction points of the beams: four large experiments called CMS, ALICE, LHCb, and ATLAS as well as three small experiments called TOTEM, MoEDAL, and LHCf.
This thesis uses data collected by the ATLAS (A Large Toroidal Lhc ApparatuS) detector.
ATLAS is a multipurpose particle detector designed to maximize acceptance and precision in order to discover and study new particles.
The detector has a cylindrical geometry that covers nearly the entire solid angle surrounding the IP.
This section will 
There are four sub-detectors that make up the ATLAS detector.

% Note: principles of detection
% \subsection{Design Philosophy}
\subsection{Coordinate System}
\subsection{Magnet System}
%magnet
The magnet system of ATLAS includes a thin superconducting solenoid surrounding the ID which generates a 2T magnetic field used for tracking particles in the ID.
Further out from the IP is a larger toroidal magnet that creates a 4T magnetic field in the region of the Muon Spectrometer.
Two separate magnets also produce a 4T field in the endcap regions.

\subsection{Inner Detector}
%ID
Closest to the IP is the Inner Detector (ID) which covers $|\eta|<2.5$ and consists of three layers of silicon pixel detector, four layers of silicon strip detector, and a transition radiation detector.
The ID provides particle tracking, momentum, identification for charged particles and jets, and signatures for both missing transverse energy and b-tagging.
The principle of detection for the silicon semiconductor detectors is the drift of electron holes through the silicon and onto a cathode. \cite{det-id}

\subsection{Calorimeters}
%LAr
Next there are two systems to provide calorimetry.
The Liquid Argon calorimeter (LAr) whose combined detectors cover $|\eta|<3.2$ provides electromagnetic calorimetry.
The LAr also provides hadronic calorimetry in the range $1.4<|\eta|<4.8$.
The principle of detection for the LAr is ionized electron drift through liquid argon to deposit charge onto an accordion configuration of anodes.
There is also a presampler to discriminate electrons from photons and compensate for energy lost through the detector.
%Tile
In addition to the LAr, an iron/scintillator calorimeter provides hadronic calorimetry in the range $|\eta|<1.7$.
The active tiles are coupled to fiber optics that carry scintillation light to PMT's for readout.
These are interspaced with iron absorbers.
\subsubsection{Electromagnetic Calorimeter}
\subsubsection{Hadronic Calorimeter}
\subsubsection{Forward Calorimeter}

\subsection{Muon System}
%Muon
The outermost ATLAS system is the Muon Spectrometer.
This system consists of trigger and precision detectors with overlapping acceptance.
Resistive Plate Chambers (RPC) in $|\eta|<1.05$ and Thin Gap Chambers (TGC) in $1.05<|\eta|<2.7$ provide the trigger for the MS.
Monitored Drift Rubes (MDT) in the barrel and endcap regions ($|\eta|<2.7$) provide precision tracking of muons along with the Cathode Strip Chambers (CSC) in the region from $2.0<|\eta|<2.7$.
Muons play an important role in this analysis, which warrants a more detailed description of the muon system.
The MDTs are single wire ($50\mu m$) drift tubes (aluminum, radius 30mm) filled with 7\% $CO_2$ and 93\% Ar gas.
These tubes are arranged in multilayers of 3 or 4 layers of drift tubes.
Multilayers are grouped together into chambers with two multilayers each.
Together, the MDTs achieve a spatial resolution of $\approx80\mu m$ and a momentum resolution of 10\% at Pt=1TeV.
The CSCs are multiwire proportional chambers with four $\eta$ layers and four $\phi$ layers of anodes.
At their peak performance, the CSC have a resolution of $\approx60\mu$m.
They are used in the high $\eta$ region where the particle rate ($>150 Hz/cm^2$) exceeds the tolerance of the MDTs.
The RPCs provide the muon trigger and $\phi$ coordinates in the barrel region ($|\eta|<2.7$).
These chambers consist of $\eta$ and $\phi$ planes of cathode and anode pads separated by a 2mm gas gap and a voltage gradient.
The TGCs fill a role similar to the RPCs in the end-cap region ($|\eta|<1.05$).
The TGCs are multiwire proportional chambers with several layers of $\eta$ and $\phi$ planes. \cite{det-muon}
\subsubsection{MDT} % precision
\subsubsection{CSC} % precision
\subsubsection{RPC}
\subsubsection{TGC}

\subsection{Trigger System}
\subsubsection{L1 Trigger}
\subsubsection{HLT Trigger}
\subsection{Data Acquisition}

%\subsection{Reconstruction}
%%particle ID and reconstruction
%The detector subsystems described above together provide the information that is used to identify particles in an event.
%The bending of tracks in the ID and the MS is used to measure a particle's charge (from the direction) and momentum (from the sagitta).
%Since electrons usually are unable to penetrate all the way to the MS, a charged track in the ID and MS is usually produced by a muon.
%Energy deposits in the calorimeters is associated with a track and corresponds to a particle's energy.
%Electromagnetic particles (electrons, muons, and photons) produce showers in the LAr.
%Strongly interacting particles (hadrons like protons and pions) produce broad showers in the hadronic calorimeter.
%The momentum and energy of a particle defines its mass and therefore the particle ID.

