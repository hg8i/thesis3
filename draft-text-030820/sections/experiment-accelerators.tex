\section{High Energy Physics at Colliders}

The study of high energy particle physics requires laboratory conditions with where high densities of energy are concentrated into a tiny volume.
One method to produce such an environment is with a particle collider.
This section discusses some of the principles that allow colliders to enable the study of otherwise unreachable aspects of our universe.

In general, a collider has the purpose of steering two beams of charged particles such that their constituent particles collide with high energy.
A number of subsystems are required to do this.
First an \emph{accelerator} must boost the beams to a high energy with the use of electric fields.
Next, a series of magnets bend and focus the beam. 
A common geometry for a collider, and one employed by the LHC, is a circular collider where two counter rotating beams are guided by dipole magnets around in a circular orbit.
Along the orbit, various magnets will focus and defocus the beam, and alter its trajectory to keep the beam close to a reference orbit.
Finally, the beams are steered to collide with each other in an \emph{interaction point}

% \subsection{Acceleration}
Accelerators are machines designed to take in particles with a given energy, and output particles with a higher energy.
A common device to achieve this is the radiofrequency (RF) cavity: a conducive cavity driven by an oscillating electric potential.
The dimensions of the cavity are selected such that the driving potential produces a resonating standing EM wave within it.
The standing wave produces an alternating electric field.
A packet of beam particles (a bunch) passing through the cavity will be accelerated by the electric field $\pmb E$, as described by the relativistic Lorentz force in Equation \ref{eqn:lorentzForce}.
\begin{equation}\begin{split}\label{eqn:lorentzForce}
\frac{d\pmb{p}}{dt}=q(\pmb E+\pmb v\cross\pmb B)
\end{split}\end{equation} 
Here $\pmb B$ is the magnetic field and $\pmb v$ is the bunch's velocity.
Particles leading or trailing the bunch are pushed back into it.

% \subsection{Magnets}

\begin{figure}[h!]
\captionsetup[subfigure]{position=b}
\centering
\subcaptionbox{Dipole magnetic field\label{fig:magnetsA}}{
\includegraphics[width=0.3\textwidth]{figures/experiment/lhc/dipoleMagnet.png}
}
\subcaptionbox{Quadrupole magnetic field\label{fig:magnetsB}}{
\includegraphics[width=0.3\textwidth]{figures/experiment/lhc/quadMagnet.png}
}
\subcaptionbox{Sextupole magnetic field\label{fig:magnetsC}}{
\includegraphics[width=0.3\textwidth]{figures/experiment/lhc/sexMagnet.png}
}
\caption{The fields of magnets used in commonly colliders. {\color{red} Replace pictures!}}
\label{fig:magnets}
\end{figure}

A number of magnets are used for a variety of purposes in a collider.
The most prevalent are dipole magnets which are used to guide the trajectory of beam around the machine.
Dipole magnets have a nearly uniform magnetic field, $\pmb B$, as illustrated in Figure \ref{fig:magnetsA}
This leads to circular motion of an incident particle with charge $q$, as described by Equation \ref{eqn:lorentzForce}.

Second to dipole magnets, quadrupole magnets are used to focus and defocus the beam profile.
An illustration of a quadrupole field is given in Figure \ref{fig:magnetsB}.
A beam passing through a quadrupole is simultaneously focused in one plane, and defocused and the perpendicular plane.
Quadrupole magnets are usually grouped together in order to provide an overall effect to the beam.
A group of two quadrupoles, the second rotated 90 degrees from the first, have the effect of focusing a beam in both planes.

A third magnet configuration is the sextupole, consisting of an arrangement of three dipoles
A sextupole is useful for adjusting momentum dependant behavior of the beam. This will be expanded on after a discussion of beam characteristics.
The field of a sextupole is illustrated in Figure \ref{fig:magnetsC}.

% \subsection{Colliding Beams}
The final task of a collider, after reaching a stable energy, is to steer the beams into collisions.
As the particles making up the beam collide, they interact and transform the incident energy into an explosion of outgoing particles.
The locations of the collisions is such that the outgoing particles can be detected by an experiment.

\section{CERN Accelerator Complex}

\begin{figure}[h!]
\captionsetup[subfigure]{position=b}
\centering
\includegraphics[width=1.0\textwidth]{figures/experiment/accelComplex-small.png}
\caption{The CERN accelerator complex - 2019 \ref{accelComplex}}
\label{fig:accelComplex}
\end{figure}

The LHC requires input beams with high intensity and energy.
It is the purpose of the LHC injector chain to provide this beam.
Four accelerators make up the chain: the Linac 2, the PS Booster, the PS, and the SPS.
The output of the chain is a proton beam of with an energy of 450~GeV.
This section describes each of these machines, and the proton beams that they produce.\cite{schindl}
The accelerator complex is illustrated schematically in Figure \ref{fig:accelComplex}.

% \subsection{Linac 2}
The first accelerator in the injector chain is the Linac 2.
Protons are sourced from a canister of hydrogen gas, and separated by a 90~keV duoplasmatron ion source\footnote{A cathode emits electrons which ionize H$_2$ gas, separating the atoms, which are then accelerated electrostatically towards an anode.}.
The protons enter a 1~m RF quadrupole and are accelerated to 750~keV.
This beam enters the Linac 2, a linear accelerator which dates to 1978.
The Linac 2 accelerates protons to 50~MeV over the course of 36~m using a series of increasingly long RF cavities. \cite{manglunki}

% \subsection{Proton Synchrotron Booster}
The beam output of the Linac 2 is transferred to the \emph{Proton Synchrotron Booster} (PSB).
The PSB, which began construction in 1968, is a circular synchrotron with a diameter of 157~m.
The incoming beam from the Linac 2 is split vertically by an electrostatic deflector to four levels.
It is at this stage that the beams are divided into \emph{bunches} of protons, separated by empty space.
These four beams enter four circular rings stacked on top of each other, quadrupling the capacity of the PSB \footnote{The PS is designed to accept 5 bunches from each PSB ring for a total of 20 bunches. For LHC operation, it accepts one bunch from each PSB ring.}. \cite{reich}
The original PSB was renovated for the LHC, and the output energy was increased from 800~MeV to 1.4~GeV. 
This helped reduce instabilities related producing denser beams.
\cite{schindl}

% \subsection{Proton Synchrotron}
The four beamlines of the PSB are recombined and extracted to the \emph{Proton Synchrotron} (PS)\footnote{Originally the CERN Proton Synchrotron (CPS).}.
The PS is a circular synchrotron with a circumference of 628~m, four times the circumference of each PSB ring. 
The PS was commissioned in 1959.
Major efforts were undertaken to prepare the PS to supply the beam for the LHC.
This included squeezing four bunches from each PSB ring into one half of the PS, and filling the PS with two PSB cycles.
Once the PS has been filled and accelerated its beams to 25~GeV, the spacing between bunches is adjusted to match the LHC's requirements.\cite{schindl}

% \subsection{Super Proton Synchrotron}
The final accelerator before the LHC is the \emph{Super Proton Synchrotron} (SPS).
The SPS was completed in 1976 with a circumference of 6.9~km.
In from 1981 to 1990 it provided beams to the UA1 and UA2 experiments, with which UA1 discovered the W and Z bosons.
As with the other CERN accelerators, the SPS underwent upgrades in order to provide beams to the LHC.
To reduce interference from the beamline, 800 vacuum pumping ports were given RF screening. \cite{schindl}
Chief among these upgrades was the construction of two new SPS-LHC transfer lines.
These lines carry two beams (clockwise and counter-clockwise) to the LHC where they are injected into the main rings.

