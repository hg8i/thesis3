\begin{center}
\textbf{\LARGE Abstract}
\end{center} 

This thesis presents a search for rare dimuon decay of the Higgs boson and a search for new physics with non-resonant phenomena at the TeV mass scale with dilepton final states.  Both studies use data, corresponding to an integrated luminosity of 139~fb$^{1}$, recorded by the ATLAS experiment in proton--proton collisions at a centre-of-mass energy of $\sqrt{s}=13$~TeV during Run~2 of the Large Hadron Collider.

The detection of the Standard Model $H\to\mu\mu$ decay is important to study the Higgs boson Yukawa couplings to the 2nd generation fermions.
Detecting the signal of Higgs boson decay to dimuon is extremely challenging due to the small decay branching fraction ($2.
2\times10^{-4}$) and large irreducible background from Drell-Yan production.
To increase the detection sensitivity, multiple event selection categories are developed based on the Higgs production modes and final state event topologies.
The major Higgs production modes at the LHC are gluon-gluon fusion (ggF), vector-boson fusion (VBF), a vector boson (W or Z) associated production (VH), and a top pair associated production (ttH).
Much of the focus of this thesis revolves around selecting VH events with different lepton final states.
The final state topologies include different jet multiplicity, and different lepton multiplicity, as well as b-jet tagging.
Machine-learning multivariate analysis (MVA) methods are developed and implemented to select events for different categories.
Background expectations are determined from the side-band events of the dimuon invariant mass spectra.
A statistic model is developed to combine events selected in different categories to extract the Higgs signal strength.
The limits set on signal production in the new phase spaces explored in this analysis are the first of their kind.
The strongest expected (observed) limit on leptonic VH production excludes signals down to 13.2(22.6) times the Standard Model prediction.
The combination of VH with the other major production modes results in a signal significance of 2.0$\sigma$.

To explore the quark and lepton internal structure, a search for new physics in the context of contact-interactions (CI) with non-resonant signals in dielectron and dimuon invariant mass spectra in the multi-TeV mass range is performed.
The experimental signal of the underlying physics would enhance the dilepton event rate at the TeV mass scale.
The major background is the dilepton from the Drell-Yan process.
A functional form is fit to the data to model the background.
This is done in a low-mass control region where the signal is expected to be negligible, while the function is extrapolated to several high-mass signal regions where an enhancement of events is expected above the background processes.
This is a novel approach at the LHC, which overcomes the limitations of both statistical and systematical uncertainties by using Monte Carlo events to estimate the background contribution.
In this search, no significant deviation in data is observed with respect to the expected background.
Upper limits on the visible cross-section times branching ratio are set in this search.
These, along with benchmark CI signal efficiencies, can be interpreted as limits in terms of a variety of signal models.
The lower limits on the energy scale of CI, $\Lambda$, reaches 35.8 TeV, indicating the quarks and leptons are still point-like particles at 10$^{-20}$ m.
These are the strongest limits on $q\bar{q}\ell^+\ell^-$ contact-interactions to date.

