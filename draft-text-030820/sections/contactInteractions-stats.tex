\section{Statistical Analysis}\label{sec:ciStat}

Statistical methods are used to distill two types of information from the collected dataset.
First, what is the probability that the observed data is incompatible with the background-only hypothesis.
Second, what is smallest putative signal such that, if extant, would produce a signal+background hypothesis that is incompatible with the observed data.
The former is answered by a significance test, described in Section \ref{sec:ciSigTest}, while the latter is answered by setting a limit, described in Section \ref{sec:ciLimitSetting}.

\subsection{CLs}

The fundamental tool used to compare two hypotheses is the \emph{test statistic}.
While this can be any quantity calculated from data, an optimal choice for the test statistic may be made to best resolve the difference between the two hypotheses.
The Neyman-Person lemma states that the likelihood ratio test is the most likely to reject the null hypotheses given the alternate hypotheses is true.
The likelihood ratio is defined:
\begin{equation}\begin{split}
\Lambda(x)=\frac{\mathcal{L}(\theta_1|x)}{\mathcal{L}(\theta_0|x)},
\end{split}\end{equation} 
where $\mathcal{L}(\theta_0|x)$ and $\mathcal{L}(\theta_1|x)$ are the likelihoods to observe data $x$ under the null and alternate hypotheses, respectively.
Data measured at larger values of $\Lambda(x)$ are \emph{less compatible} with the background hypothesis.

The PDF of $\Lambda(x)$ is defined under both the null and alternate hypotheses.
Taking first the PDF under the null hypothesis, $\Lambda_0(x)$.
The integral of the test statistic $\Lambda_0(x)$ above a given observed value of $x$, $x_\text{obs}$, defines the \emph{p-value} $p_0$ of the observation.
This is the probability to observe a value of $x$ that is \emph{less} compatible with the null hypothesis than the observed value.
The complement of the p-value, calculated under the null hypothesis, defines the value $\clb\equiv1-p_0$.
An analogous value, $\clsb$, is defined for the likelihood ratio under the alternative hypothesis, $\Lambda_1(x)$.
For a measured $x_\text{obs}$, $p_1$ is the integral of $\Lambda_1(x)$ above that point, and $\clsb\equiv1-p_1$
Finally the ratio of these two values defines $\cls\equiv \clsb/\clb$.
The motivation 

\subsection{Statistical Model}

Each statistical question is answered through the comparison of null and alternate hypotheses.
% The 


\subsection{Significance test}\label{sec:ciSigTest}

A hypothesis test is performed in each of the four signal regions of the analysis.
The null hypothesis predicts the number of background events in the signal region, using the integral of the extrapolated fitted background-only functional form (Equations \ref{eqn:ciBkgEe} and \ref{eqn:ciBkgMm}).
The alternative hypothesis predicts the same number of background events as the null hypothesis, with the addition of a number of signal events.
The alternative hypothesis is fit to the observed data.


\subsection{Limit test}\label{sec:ciLimitSetting}
