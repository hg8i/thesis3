\subsection{Physics with Lagrangians}

The previous section dealt with the abstract notions of groups and relations.
Representations of these groups to provide a useful description of the physical world.
This section will discuss how these representations used with physics.
The task is to build a map between the mathematical structures, which are stripped of physical meaning, and the physical model.
This is done through the framework of Quantum Field Theory (QFT).

The connection between the abstract words of groups and symmetries, and the more concrete world of QFT is as follows.
The fields of fermions correspond to the modules of certain representations, on which the members of the algebra can act.
Charges, the physical characteristics of particles that are conserved, correspond to the eigenvalues of the group's generators.
The number of charges described by a group is equal to the dimension of the representation.
Physical interactions through a force are described by the eigenvectors of the earlier defined Cartan generators of the group.
The generators of the group describe particle that mediate forces to particles charged under the group.
The force carrying particle described by the Cartan generators do not change corresponding charge of a particle.
Non-Cartan generators however can change charge.

\subsubsection{Hamilton's Principle}\label{sec:lagrangian}

Hamilton's principle of least action considers a system defined by a field, $\phi$, and its four dimensional derivatives $\partial_\mu\phi$.
The state evolves between two points in spacetime, $x_0$ and $x_1$.
The \emph{action} is a functional of the path taken by $\phi$ between $x_0$ and $x_1$:
\begin{equation}\begin{split}
S[\phi]=\int_{x_0}^{x^1} d^4x\mathcal{L}(\phi,\partial_\mu\phi).
\end{split}\end{equation}
The action is defined as a path integral between these points of a function of the path, $\mathcal{L}(\phi,\partial_\mu\phi)$.
$\mathcal{L}$ is a functional of the full path, and is called the \emph{Lagrangian density}, often shortened to a \emph{Lagrangian}.
Hamilton's principle states that action $S$ is minimized when the fields $\phi_i(x)$ follow their equation of motion.
This minimization results in the Euler-Lagrange equation:
\begin{equation}\begin{split}\label{eqn:el}
    % \frac{\partial\mathcal{L}}{\partial\phi}-\partial_\mu\left(\frac{\partial\mathcal{L}}{\partial(\delta_\mu\phi)}\right)=0
    \frac{\partial\mathcal{L}}{\partial\phi_i}-\partial_\mu\left(\frac{\partial\mathcal{L}}{\partial(\delta_\mu\phi_i)}\right)=0
\end{split}\end{equation} 
Solving Equation \ref{eqn:el} for a given Lagrangian density yields the \emph{equations of motion} for each field.

% Fields
This raises the issue of what the fields $\phi(x)$ are.
In QFT, \emph{operators} are introduced for each point in spacetime $x$.
These operators, $\phi$, are labeled by their corresponding point as $\phi(x)$.
As in quantum mechanics operators can intuitively thought of as representing a measurement. In this case, they represent a measurement of a corresponding field strength.
The operators act on \emph{state} vectors that represent the physical system.
For example, the operator $\phi$ can act on the vector representing the vacuum state, $\ket{0}$, at location $x$. This is denoted as $\phi(x)\ket{0}$.
The behavior of fields is described by the Lagrangian.

A commonly used type of field is the \emph{scalar} field.
Scalar fields are scalar valued, having a single value at all spacetime points $x$.
The Higgs boson is an example of particles represented by a scalar field.
Another commonly used type of field is the \emph{Dirac} field.
These fields are four-component, represented by 4-vectors.
Fermions are an example of particles represented by Dirac fields.
Finally,a commonly used type of field is the \emph{vector} field.
An example is the 4-vector from electromagnetism $A^\mu=(V,\vec{A})$ consisting of the electric potential $V$ and the magnetic potential $\vec{A}$.
In the following paragraph, Lagrangians containing each of these fields are shown, along with the corresponding result of Equation \label{eqn:el}.


The equations of motion for each field is derived from the Lagrangian.
As an example, a scalar field $\psi$ described by a simple Lagrangian given in Equation \ref{eqn:scalarLagrangian}.
\begin{equation}\begin{split}\label{eqn:scalarLagrangian}
    \mathcal{L}_\text{scalar}=&\frac{1}{2}\partial_\mu\phi\partial^\mu\phi-\frac{1}{2}m^2\phi^2 \\
% \frac{\partial\mathcal{L}}{\partial\phi}=&-m^2\phi \\
% \partial_\mu\frac{\partial\mathcal{L}}{\partial(\delta_\mu)}=&\partial_\mu\partial^\mu\phi \\
% \partial_\mu\partial^\mu\phi+m^2\phi=&0; \quad\text{Klein-Gordon wave equation}
\end{split}\end{equation}
Plugging equation Equation \label{eqn:scalarLagrangian} into the Euler-Lagrange Equation \label{eqn:el} yields the Klein-Gordon equation of motion.
An analogous Lagrangian can be written for Dirac fields, $\Psi$, and is given in Equation \ref{eqn:diracLagrangian}.
\begin{equation}\begin{split}\label{eqn:diracLagrangian}
    \mathcal{L}_\text{Dirac}=i\overline{\Psi}\gamma^\mu\partial_\mu\Psi-m\overline{\Psi}\Psi \\
\end{split}\end{equation} 
Where $\overline{\Psi}=\Psi^\dagger\gamma^0$ is the Hermitian conjugate.
Taking the derivatives in Equation \label{eqn:el} yields the Dirac equation of motion.
Finally a Lagrangian for the vector field $A^\mu$ can be written, with the definition $F^{\mu\nu}\equiv\partial^\mu A^\nu-\partial^\nu A^\mu$, as shown in Equation \ref{eqn:vectorLagrangian}.
\begin{equation}\begin{split}\label{eqn:vectorLagrangian}
    \mathcal{L}_\text{vector}=&-\frac{1}{4}F_{\mu\nu}F^{\mu\nu} \\
\end{split}\end{equation} 
Which yields the equation of motion is identical to Maxwell's equations with zero electric current.
The fields and their equation of motion are summarized in Table \ref{tab:fields}.

\begin{table}[htp]
\begin{center}
{\footnotesize
\begin{tabular}{l | l l l l}
\toprule
Field & Symbol & Equation of motion  \\
Scalar & $\phi(x)$    & $\partial_\mu\partial^\mu\phi+m^2\phi=0$  \\
Dirac  & $\Psi(x)$    & $i\gamma^\mu\partial_\mu\Psi-m\Psi=0$  \\
Vector & $A^\mu(x)$   & $\partial_\mu F^{\mu\nu}=0$  \\
\multirow{2}{*}{Weyl} & \multirow{2}{*}{$\psi(x)$}   & $i\overline{\sigma}^\mu\partial_\mu\psi_L=0$ \\
                      &                              & $i\sigma^\mu\partial_\mu\psi_R=0$  \\
\midrule
\bottomrule
\end{tabular}
}
\caption{Summary fields and their transformations. In some cases, additional symbols will be used to avoid ambiguity.}
\label{tab:fields}
\end{center}
\end{table}

The Lagrangians of Equations \ref{eqn:scalarLagrangian}, \ref{eqn:diracLagrangian}, and \ref{eqn:vectorLagrangian} all describe freely propagating particles; scalar, Dirac, and vector particles respectively.
Interactions between particles complicate the Lagrangian with additional terms containing multiple fields in a single term.

\subsubsection{The Matrix Element}

The Lagrangian formalism set out in Section \ref{sec:lagrangian} allows the definition of a model.
The next step is to extract the observable predictions of the model.
Nearly all observations of particles relate to the question: ``given some initial state, what is the probability of observing a final state''.
In a particle collider, the initial state may be the colliding beams, and the final state may include some number of electrons.
The task is to predict the probability of the initial state to evolve to the final state, given the dynamics of a given model.
This probability is labeled $\mathcal{P}_{i\to f}$.
In quantum mechanics, the time evolution of a state $\ket{\Phi}$ is given by the time dependant Schrodinger equation,
\begin{equation}\begin{split}\label{eqn:schrodinger}
i\hbar\frac{d}{dt}\ket{\Psi}=H\ket{\Psi},
\end{split}\end{equation} 
where the energy of the system is described by $H$, its Hamiltonian operator.
The Hamiltonian is derived from a given Lagrangian through a process called \emph{canonical quantization}. The details of this procedure are given in Appendix \label{sec:canQuant}.
To find the probability that an initial state with two particles $\ket{p_1;p_2}$ to evolve into a final state with multiple particles $\ket{k_1;...;k_n}$, the projection of the latter onto the former is calculated in the distant future.
\begin{equation}\begin{split}\label{eqn:ifProj}
\braket{i|f}=\braket{k_1;...;k_n|p_1;p_2}|_{t=\infty}=&_{t=\infty}|\bra{k_1;...;k_n}e^{itH}\ket{p_1;p_2}|_{t=-\infty} \\
=&|\mathcal{M}|(2\pi)^4\delta^{(4)}(p_1+p_2-\sum^n_{i=1} k_i) \\
\end{split}\end{equation} 
Here, the complexities of the Hamiltonian's action on the initial state $\ket{p_1;p_2}$ has been conveniently rolled into the \emph{matrix element}, $|\mathcal{M}|$.

Equation \ref{eqn:ifProj} shows a projection of one state onto another.
The proper probability to evolve from the initial state to the final state must be normalized.
Again, the canonical quantization is used to calculate $\braket{i|i}=4E_{p_1}E_{p_2}V^2$ and $\braket{f|f}=\prod_{i=1}^n2E_{k_i}V$ where $V$ is the volume of the box in which the experiment takes pace.
The full probability is
\begin{equation}\begin{split}\label{eqn:ifProb}
    \mathcal{P}_{i\to f}=\frac{\braket{f|i}^2}{\braket{i|i}\braket{f|f}}=\frac{|\mathcal{M}|^2(2\pi)^4TV\delta^{(4)}(p_1+p_2-\sum^n_{i=1} k_i)}{4E_{p_1}E_{p_2}V^2\prod_{i=1}^n[2E_{k_i}V]}.
\end{split}\end{equation} 
In practice it is difficult to arrange for a single interaction, such as the one presented in Equation \ref{eqn:ifProj}.
Additionally, the volume $V$ is poorly defined for an experiment.

Instead, beams of $N_b$ particles spread out over area $A$ are made to collide.
The total number of collisions $N$ then defines an effective cross-sectional area $\sigma$ for the interaction is
\begin{equation}\begin{split}\label{eqn:nCollisions}
    N=\frac{N_b^2\sigma}{A}=\sigma\int Ldt=N^2\mathcal{P}_{i\to f}
\end{split}\end{equation} 
The second and third terms define the instantaneous luminosity $L\equiv N_b^2/A$ as the number of potential collisions per time, per area.
The cross-section $\sigma$ is commonly cited in convenient units of \emph{barns}, with an exact definition of $\text{b}\equiv10^{-24}\text{cm}^2$.
The barn was named during the Manhattan Project by Marshall Holloway and C. P. Baker after the two rejected the idea of naming the unit ``John'' after John Manley.
Instantaneous luminosity benefits from no such useful shorthand, and is commonly cited in units of $\text{s}^{-1}\text{cm}^{-2}$.

The cross-section is replaces $\mathcal{P}_{i\to f}$ as the measurable prediction of the model.
The fourth term in Equation \ref{eqn:nCollisions} connects the cross-section back to the subject of the matrix element $|\mathcal{M}|$.
It is useful to calculate the differential contribution cross-section $d\sigma$ given in Equation \ref{eqn:diffCrossSection}.
\begin{equation}\begin{split}\label{eqn:diffCrossSection}
d\sigma=&\frac{|\mathcal{M}|^2}{4E_{p_1}E_{p_2}|\vec{p}_1||\vec{p}_1|}d\Phi_n \\
d\Phi_n=&(2\pi)^2\delta^{(4)}\left(p_1+p_2-\sum^n_{i=1}k_i\right)\prod^n_{i=1}\left[\frac{d^3\vec{k}_i}{(2\pi)^32E_i}\right] \\
\end{split}\end{equation} 
The volume terms in Equation \ref{eqn:ifProb}, which arose from the integrating the projections of states over all space, have canceled with volume terms that arose from integrating $\mathcal{P}_{i\to f}$ over all space.
Equation \ref{eqn:diffCrossSection} defines $d\sigma$ in terms of the differential phase space $d\Phi_n$. 
In simple $2\to2$ processes with center of mass energy $E'$, the differential phase space is greatly simplified as $d\Phi_2=\frac{k_1}{(4\pi)^2E'}d\phi d(\cos\theta)$.
